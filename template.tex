%%%%%%%%%%%%%%%%%%
% ALTACV — ATS-READY TEMPLATE
% (Single-column with faux 2-column Experience)
%
% Author: Isaac B. Michael (isaacbmichael.github.io)
% Derived from: The excellent altacv class/template by LianTze Lim.
%   This file uses the altacv class; refer to the altacv repository for the
%   class license and attribution details. This template adds an ATS mode,
%   faux 2-column right rail, icon-silent ActualText wrappers, dash normalization,
%   and alignment helpers (ExpSubRole / ExpRightRail), plus usage notes.
%
% License for this template file: MIT (see LICENSE in the repo). The altacv
% class itself remains under its original license per the upstream project.
%
% Compile
% • pdfLaTeX (portable): `latexmk -pdf template.tex`
% • XeTeX/LuaTeX: if system fonts (Roboto Slab, Lato) are installed.
%
% Quick edits
% • Template variables (name, contact, tagline, photo path)
% • ATS/Logo toggles
% • Experience using ExpEntry blocks; sub-roles via ExpSubRole or ExpRightRail
% • Technical Skills (templated for both ATS and Showcase modes)
%%%%%%%%%%%%%%%%%%

\documentclass[10pt,a4paper,ragged2e,withhyper]{altacv}

% --------------------------- Layout ---------------------------
\geometry{left=1.25cm,right=1.25cm,top=1.5cm,bottom=1.5cm,columnsep=1.2cm}

% --------------------- Columns & lists infra ------------------
\usepackage{paracol}
\usepackage{enumitem}
\usepackage{etoolbox} % for \AtBeginEnvironment

% ------------------- Engine/encoding guards -------------------
\usepackage{iftex}
\ifPDFTeX
  \usepackage[utf8]{inputenc}
  \usepackage[T1]{fontenc}
  \usepackage[rm]{roboto}
  \usepackage[defaultsans]{lato}
\else
  \setmainfont{Roboto Slab}
  \setsansfont{Lato}
\fi
\renewcommand{\familydefault}{\sfdefault}

% -------------------------- Colours --------------------------
\definecolor{SlateGrey}{HTML}{2E2E2E}
\definecolor{LightGrey}{HTML}{666666}
\definecolor{ModernCVBlue}{HTML}{1F497D}
\definecolor{LightBlue}{HTML}{4F81BD}
\definecolor{CoolGray}{HTML}{A6A6A6}

% ModernCV Blue scheme
\colorlet{name}{SlateGrey}
\colorlet{tagline}{LightBlue}
\colorlet{heading}{ModernCVBlue}
\colorlet{headingrule}{CoolGray}
\colorlet{subheading}{LightBlue}
\colorlet{accent}{LightBlue}
\colorlet{emphasis}{SlateGrey}
\colorlet{body}{LightGrey}

% ------------------------ Typography -------------------------
\renewcommand{\namefont}{\Huge\bfseries\sffamily}
\renewcommand{\personalinfofont}{\scriptsize}
\renewcommand{\cvsectionfont}{\Large\bfseries\sffamily}
\renewcommand{\cvsubsectionfont}{\normalsize\bfseries\sffamily}

% Remove the extra \smallskip inside \cvsubsection
\makeatletter
\let\IM@cvsubsection\cvsubsection
\renewcommand{\cvsubsection}[1]{%
  {\setlength{\smallskipamount}{0pt plus 0pt minus 0pt}%
   \IM@cvsubsection{#1}}%
}
\makeatother

% Org name size control used by ExpEntry (default = \normalsize\bfseries)
\newcommand{\OrgNameFont}{\normalsize\bfseries}

% Micro-typography + leniency
\usepackage[final]{microtype}
\emergencystretch=3em

% ----------------- Accessibility / ATS helpers ----------------
\usepackage{accsupp}

% Visible content with custom extracted text
\newcommand{\ATSVisual}[2]{\BeginAccSupp{ActualText={#1}}#2\EndAccSupp{}}
% Alias for fields (same behavior)
\newcommand{\ATSField}[2]{\ATSVisual{#1}{#2}}

% Executive Impact helpers
\newcommand{\impactlabel}[2]{\textbf{\textcolor{SlateGrey}{#1}}: #2}

% ATS-safe square bullet (used in Strategic Impact list)
\newcommand{\ImpactBullet}[1]{%
  \makebox[0.75em][c]{%
    \raisebox{0.25ex}{%
      \ATSVisual{- }{{\color{LightBlue}\tiny #1}}%
    }%
  }%
}

% Global list defaults
\setlist[itemize]{nosep}
\setlist{after=\vspace{0pt}} % keep lists tight

% Bullet geometry for Experience lists
\newlength{\ExLabelW}   \setlength{\ExLabelW}{0.60em}
\newlength{\ExLabelSep} \setlength{\ExLabelSep}{0.35em}

% Apply only INSIDE ExpEntry
\AtBeginEnvironment{ExpEntry}{%
  \setlist[itemize]{%
    nosep,
    label=\ATSVisual{- }{\textbullet},
    labelwidth=\ExLabelW,
    labelsep=\ExLabelSep,
    labelindent=0pt,
    leftmargin=\dimexpr\ExLabelW+\ExLabelSep\relax,
    itemindent=0pt,
    listparindent=0pt
  }%
}

% ---------------- ATS dash normalization (SAFE) ---------------
\newcommand{\ATSENdash}{\BeginAccSupp{ActualText=-}\textendash\EndAccSupp{}}
\newcommand{\ATSEMdash}{\BeginAccSupp{ActualText=--}\textemdash\EndAccSupp{}}

\ifPDFTeX
  % pdfLaTeX: map Unicode U+2013/U+2014 input to the safe macros above
  \DeclareUnicodeCharacter{2013}{\ATSENdash} % en dash → "-"
  \DeclareUnicodeCharacter{2014}{\ATSEMdash} % em dash → "--"
\else
  % Xe/LuaLaTeX: same idea via newunicodechar
  \usepackage{newunicodechar}
  \newunicodechar{–}{\ATSENdash}
  \newunicodechar{—}{\ATSEMdash}
\fi

% Spacing helper between sub-roles
\newcommand{\SubRoleGap}{\par\vspace{\medskipamount}}

% -------------------- Education line macro --------------------
\newlength{\EduYearWidth}
\setlength{\EduYearWidth}{0.07\textwidth}
\newlength{\EduLineSep}
\setlength{\EduLineSep}{0.42em}
\newcommand{\EduLine}[4]{%
  {\begingroup
    \noindent
    \makebox[\EduYearWidth][l]{\normalsize #1}%
    {\textcolor{name}{\normalsize\textbf{#2, }}\textcolor{LightBlue}{\normalsize\textbf{#3}}}%
    \hfill {\small\textnormal{\textcolor{name}{#4}}}%
    \par\vspace{\EduLineSep}%
  \endgroup}%
}

% -------------- ATS: hyphenation & Unicode extract -------------
\usepackage[none]{hyphenat}
\hyphenpenalty=10000
\exhyphenpenalty=10000

\ifPDFTeX
  \IfFileExists{glyphtounicode.tex}{\input{glyphtounicode.tex}}{}
  \pdfgentounicode=1
\fi

% ----- Faux 2-column Experience (year gutter + right rail) -----
% Match paracol's 0.07 : 0.93 split with the same \columnsep (1.2cm)
\newlength{\ExpWorkW}  \setlength{\ExpWorkW}{\dimexpr\textwidth - \columnsep\relax}
\newlength{\ExpYearW}  \setlength{\ExpYearW}{0.07\ExpWorkW}
\newlength{\ExpSep}    \setlength{\ExpSep}{\columnsep}
\newlength{\ExpRightW} \setlength{\ExpRightW}{\dimexpr\ExpWorkW - \ExpYearW\relax}

% Roomier vertical spacing between lines in each entry
\newcommand{\ExpGapA}{\vspace{0.25em}}  % company → job line
\newcommand{\ExpGapB}{\vspace{0.35em}}  % job line → summary

% One-column Experience entry (minipage right rail)
% Args: 1=Years, 2=Company, 3=Title, 4=Dept/tag (may be empty), 5=Location
\newenvironment{ExpEntry}[5]{%
  \par\noindent
  \makebox[\ExpYearW][l]{\normalsize\textcolor{body}{#1}}%
  \hspace{\ExpSep}%
  \begin{minipage}[t]{\ExpRightW}
    \raggedright
    {\textcolor{subheading}{{\OrgNameFont #2}}}\par
    \ExpGapA
    \noindent
    {\textcolor{name}{\normalsize\bfseries #3}}%
    \if\relax\detokenize{#4}\relax\else{\textcolor{name}{\normalsize,\ #4}}\fi
    \hfill{\small\textcolor{name}{#5}}\par
    \ExpGapB
}{%
  \end{minipage}\par
}

% Sub-role within the same entry
\newcommand{\ExpSubRole}[3]{%
  \par\noindent
  {\textcolor{name}{\normalsize\bfseries #1}}%
  \if\relax\detokenize{#2}\relax\else{\textcolor{name}{\normalsize,\ #2}}\fi
  \hfill{\small\textcolor{name}{#3}}\par
  \ExpGapB
}

% --- Right-rail wrapper for content outside ExpEntry (keeps alignment)
%     Use when you want a sub-role *after* an ExpEntry but aligned with the
%     same left edge as the role/title. Also mirrors the ExpEntry list style.
\newenvironment{ExpRightRail}{%
  \par\noindent
  \makebox[\ExpYearW][l]{}%
  \hspace{\ExpSep}%
  \begin{minipage}[t]{\ExpRightW}
    \raggedright
    % Mirror the ExpEntry itemize style so bullets align
    \setlist[itemize]{%
      nosep,
      label=\ATSVisual{- }{\textbullet},
      labelwidth=\ExLabelW,
      labelsep=\ExLabelSep,
      labelindent=0pt,
      leftmargin=\dimexpr\ExLabelW+\ExLabelSep\relax,
      itemindent=0pt,
      listparindent=0pt
    }%
}{%
  \end{minipage}\par
}

% ------------------------- Global toggles ----------------------
% How to use:
% • ATS mode: Uncomment \ATStrue to hide the photo, switch skills to a
%   single-column list, and normalize dashes in the text layer.
% • Logo: \LOGOtrue enables an optional photo in showcase mode; ATS mode
%   forces it off automatically.
% • Tip: Escape special characters in header fields (\&, \%, \#, \_).
% • Need a sub-role after an entry? Wrap the block in ExpRightRail to share
%   the same right-rail alignment and bullet geometry.
\newif\ifATS              % ATS mode: set true to hide photo, switch skills to 1-col, etc.
% \ATStrue                 % ← uncomment to enable ATS mode

\newif\ifLOGO             % independent logo toggle (ATS forces it off)
\LOGOtrue
\ifATS\LOGOfalse\fi

% ----------------------- Contact system ------------------------
% 1) Core line builder
\newcommand{\ContactLineWithMark}[3]{%
  % #1 visible mark (e.g., \textbullet\ )
  % #2 ATS text (e.g., "Email: ...")
  % #3 visible content (e.g., \href{...}{...})
  \ATSVisual{#2}{\textcolor{LightBlue}{#1}\hspace{-0.25em}#3\hspace{2.25em}}%
}

% 2) Default marks per method (FontAwesome via altacv)
\newcommand{\MarkEmail}{\faEnvelope\ }
\newcommand{\MarkPhone}{\faPhone\ }
\newcommand{\MarkLinkedIn}{\faLinkedin\ }
\newcommand{\MarkPortfolio}{\faGlobe\ }
\newcommand{\MarkLocation}{\faMapMarker\ }

% 3) Wrappers with optional per-call override
\newcommand{\EmailLine}[2][]{%
  \ContactLineWithMark{\if\relax\detokenize{#1}\relax \MarkEmail \else #1 \fi}{Email: #2 |}{\href{mailto:#2}{#2}}%
}
\newcommand{\PhoneLine}[2][]{%
  \ContactLineWithMark{\if\relax\detokenize{#1}\relax \MarkPhone \else #1 \fi}{Phone: #2 |}{\href{tel:#2}{#2}}%
}
% #2 should be like "linkedin.com/in/<handle>"
\newcommand{\LinkedInLine}[2][]{%
  \ContactLineWithMark{\if\relax\detokenize{#1}\relax \MarkLinkedIn \else #1 \fi}{LinkedIn: linkedin.com/in/#2 |}{\href{https://linkedin.com/in/#2}{#2}}%
}
% #2 e.g., "example.com"
\newcommand{\PortfolioLine}[2][]{%
  \ContactLineWithMark{\if\relax\detokenize{#1}\relax \MarkPortfolio \else #1 \fi}{Portfolio: #2 |}{\href{https://#2}{#2}}%
}
% #2 visible name (e.g., "City, ST"), #3 URL
\newcommand{\LocationLine}[3][]{%
  \ContactLineWithMark{\if\relax\detokenize{#1}\relax \MarkLocation \else #1 \fi}{Location: #2}{\href{#3}{#2}}%
}

% ---------------------- Metadata & PDF properties ------------------
% NOTE: altacv already loads hyperref (via 'withhyper').
% If you want to set PDF metadata, UNCOMMENT the block below.
% To avoid "Option ... has already been used" warnings, keep these settings
% in one place only (here) and do NOT also pass them as class/package options.
%
% \hypersetup{
%   pdftitle   = {ATS-Ready altacv Resume Template},
%   pdfauthor  = {Isaac B. Michael},
%   pdfsubject = {ATS-friendly LaTeX resume template derived from AltaCV},
%   pdfkeywords= {LaTeX,resume,CV,ATS,altacv}
% }

% ---------------------- Template variables ---------------------
\newcommand{\FullName}{Your Name}
\newcommand{\RoleTagline}{Your Role Tagline (e.g., Data \& Services Leader)} % NOTE: escape ampersands as \&
% Contact
\newcommand{\Email}{you@example.com}
\newcommand{\Phone}{000-000-0000}
\newcommand{\LinkedInHandle}{your-handle}
\newcommand{\PortfolioURL}{example.com}
\newcommand{\LocationName}{City, ST}
\newcommand{\LocationURL}{https://www.google.com/maps/search/?api=1&query=City}
% Assets
\newcommand{\PhotoPath}{globe_high.png} % replace with your image (optional)

% ============================= BODY ============================
\begin{document}

\sloppy

\name{\FullName}
\tagline{\RoleTagline}

% ---- Logo (ATS-silent) ----
\ifLOGO
  \ifATS\else
    \BeginAccSupp{ActualText={}}%
      \photoR{1.9cm}{\PhotoPath}%
    \EndAccSupp{}%
  \fi
\fi

% --- Header with fields (custom per-method marks) ---
\personalinfo{%
  \EmailLine{\Email}%
  \PhoneLine{\Phone}%
  \LinkedInLine{\LinkedInHandle}%
  \PortfolioLine{\PortfolioURL}%
  \LocationLine{\LocationName}{\LocationURL}%
}

\makecvheader

% ========================= EXEC PROFILE ========================
\cvsection{Professional Summary}
\begin{quote}
\normalsize{\textbf{One-sentence value proposition} highlighting scope (teams, budgets, programs) and how you make outcomes repeatable via automation, governance, and design. Keep it concise and metrics-aware.}
\end{quote}

% ======================== STRATEGIC IMPACT =====================
\cvsection{Key Achievements}
\begin{itemize}[label={},labelsep=0.35em,leftmargin=1.55em,itemsep=0.5mm]
  \item[\ImpactBullet{\faSquare}]
    \impactlabel{Policy Adoption \& Standards}{Example: Co-authored guidance adopted by a federal body; replaced outdated thresholds with quality-focused evaluation.}
  \item[\ImpactBullet{\faSquare}]
    \impactlabel{Governance \& Controls}{Instituted enterprise KPI scorecards and codified SLAs/runbooks with escalation paths to improve reproducibility and timeliness.}
  \item[\ImpactBullet{\faSquare}]
    \impactlabel{Commercial Model}{Built profit-sharing SOWs/MOUs with milestones and reporting cadence—establishing durable margins for services.}
  \item[\ImpactBullet{\faSquare}]
    \impactlabel{Systems Architecture}{Designed a robust analytics/reporting pipeline delivering scalable, reproducible outputs for non-technical users.}
\end{itemize}

% =========================== EXPERIENCE ========================
% Usage:
% • Duplicate ExpEntry blocks for each role.
% • Put sub-roles INSIDE the same ExpEntry with \ExpSubRole for perfect alignment.
% • If you must add sub-role content AFTER the ExpEntry, wrap it in:
%   \begin{ExpRightRail} ... \end{ExpRightRail}
\cvsection{Experience}

% ---- Template Entry 1 ----
\begin{ExpEntry}{20XX\ATSENdash Present}{\ATSField{Company}{Company Name}}{\ATSField{Title}{Role Title}}{Department or Business Unit}{City, ST}
\textbf{Summary:} \textcolor{body}{One-to-two sentence summary: scope, scale, tech stack, business outcomes.}

\smallskip
\begin{itemize}
  \setlength{\itemsep}{0.5mm}
  \item \textbf{Service Design \& Automation:} What you standardized or automated; include success metrics when possible.
  \item \textbf{Validation \& Parity:} How you ensured reproducibility and accuracy (QA, audit logs, parity checks).
  \item \textbf{Governance \& Enablement:} Playbooks, templates, SLAs, escalation paths; how they improved delivery.
\end{itemize}
\end{ExpEntry}

\medskip

% ---- Template Entry 2 ----
\begin{ExpEntry}{20XX\ATSENdash 20XX}{Organization}{Role Title}{Team / Program}{City, ST}
\textbf{Summary:} \textcolor{body}{Impact-focused summary: the problem, your solution, the measurable result.}

\smallskip
\begin{itemize}
  \setlength{\itemsep}{0.5mm}
  \item \textbf{Roadmap ownership:} Owned roadmap and acceptance criteria; drove cross-functional delivery.
  \item \textbf{Data modernization:} Platform shifts, automation, or re-architecture that improved maintainability and reliability.
  \item \textbf{Stakeholder leadership:} Briefings, trainings, and partner management that raised quality and timeliness.
\end{itemize}
\end{ExpEntry}

\medskip

% ---- Optional Sub-role within same organization ----
% Preferred: place sub-roles *inside* the ExpEntry above. If you need
% them outside, wrap in ExpRightRail to preserve alignment:
\begin{ExpRightRail}
\ExpSubRole{Sub-Role Title}{Group / Team}{City, ST}
\textbf{Summary:} \textcolor{body}{Optional: leadership of a seasonal or parallel program; include scale and results.}

\begin{itemize}
  \setlength{\itemsep}{0.5mm}
  \item \textbf{Scale \& Delivery:} e.g., Directed N sections serving X students; implemented cross-section comparability.
  \item \textbf{Learning Impact:} Outcome metrics (pre\ATSENdash post, pass rates, CSAT/NPS).
  \item \textbf{Operations:} SOPs, grading standards, automation, vendor integrations.
\end{itemize}
\end{ExpRightRail}

% ========================= TECHNICAL SKILLS ====================
% ATS vs Showcase:
% • ATS branch: single-column, product-agnostic bullets (parser-friendly).
% • Showcase branch: 3 columns; keep generic placeholders if publishing publicly.
\cvsection{Technical Skills}

\ifATS
% --- Single-column, ATS-friendly list ---
\begin{itemize}[leftmargin=*, itemsep=0.2em,
                label=\ATSVisual{- }{\textbullet}, labelsep=0.35em]
  \item \textbf{Data \& Automation:} Languages and tools (programming, scripting, version control); CI/CD; containers/orchestration
  \item \textbf{Methods \& Governance:} Methods and frameworks; QA/testing; KPI scorecards; SLAs/runbooks; escalation paths
  \item \textbf{Cloud \& Analytics:} Cloud platforms/data warehousing; ETL/ELT; dashboards/BI
\end{itemize}
\else
% --- Human-friendly 3-column layout ---
\columnratio{0.3,0.4,0.3}
\begin{paracol}{3}

\cvsubsection{Data \& Automation}
\footnotesize
Languages and tools; version control; CI/CD; containers and orchestration

\switchcolumn

\cvsubsection{Methods \& Governance}
\footnotesize
Methods and frameworks; QA/testing; KPI scorecards; SLAs and runbooks; escalation paths

\switchcolumn

\cvsubsection{Cloud \& Analytics}
\footnotesize
Cloud platforms and data warehousing; ETL/ELT; dashboards and BI

\end{paracol}
\fi

% ============================= EDUCATION =======================
\cvsection{Education}

\EduLine{20XX}{Degree or Program}{University or Institution}{City, ST}
\EduLine{20XX}{Degree or Program}{University or Institution}{City, ST}

\divider

\end{document}
